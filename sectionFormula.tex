%%%%%%%%%%%%%%%%%%%%%%%%%%%%%%%%%%%%%%%%%%%%%%%%%%%%%%%%%%%%%%%%
% Form
%%%%%%%%%%%%%%%%%%%%%%%%%%%%%%%%%%%%%%%%%%%%%%%%%%%%%%%%%%%%%%%%
\section{基本公式推导}
下面推列举一些需要用到的公式定理。

% ////////////////////////////////////////
\subsection{改进春分点轨道根数}
改进的春分点轨道与开普勒轨道根数之间的转换关系定义如下\cite{mxubo2016}:
\begin{align}
    \begin{aligned}
        p =& a\left(1-e^{2}\right) \\
        f =& e \cos (\omega+\Omega) \\
        g =& e \sin (\omega+\Omega) \\
        h =& \tan (i / 2) \cos \Omega \\
        k =& \tan (i / 2) \sin \Omega \\
        L =& \Omega+\omega+\theta
    \end{aligned} \label{eqFormEquinoctial1}
\end{align}
\begin{align}
    \begin{aligned}
        a =& \frac{p}{1-f^{2}-g^{2}} \\
        e =& \sqrt{f^{2}+g^{2}} \\
        i =& 2 \arctan \sqrt{h^{2}+k^{2}} \\
        \Omega =& \arctan \left(\frac{k}{h}\right) \\
        \omega =& \arctan \left(\frac{g}{f}\right)-\arctan \left(\frac{k}{h}\right) \\
        \theta =& L-\arctan \left(\frac{g}{f}\right)
    \end{aligned} \label{eqFormEquinoctial2}
\end{align}
式\eqref{eqFormEquinoctial1}\eqref{eqFormEquinoctial2}分别表示。

