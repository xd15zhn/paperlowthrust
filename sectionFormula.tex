%%%%%%%%%%%%%%%%%%%%%%%%%%%%%%%%%%%%%%%%%%%%%%%%%%%%%%%%%%%%%%%%
% Form
%%%%%%%%%%%%%%%%%%%%%%%%%%%%%%%%%%%%%%%%%%%%%%%%%%%%%%%%%%%%%%%%
\section{基本公式}
下面推列举一些需要用到的公式定理。

% ////////////////////////////////////////
\subsection{改进春分点轨道根数}
改进的春分点轨道与开普勒轨道根数之间的转换关系定义如下\cite{mxubo2016}:
\begin{align}
    \begin{aligned}
        p =& a\left(1-e^{2}\right) \\
        f =& e \cos (\omega+\Omega) \\
        g =& e \sin (\omega+\Omega) \\
        h =& \tan (i / 2) \cos \Omega \\
        k =& \tan (i / 2) \sin \Omega \\
        L =& \Omega+\omega+\theta
    \end{aligned} \label{eqFormEquinoctial1}
\end{align}
\begin{align}
    \begin{aligned}
        a =& \frac{p}{1-f^{2}-g^{2}} \\
        e =& \sqrt{f^{2}+g^{2}} \\
        i =& 2 \arctan \sqrt{h^{2}+k^{2}} \\
        \Omega =& \arctan \left(\frac{k}{h}\right) \\
        \omega =& \arctan \left(\frac{g}{f}\right)-\arctan \left(\frac{k}{h}\right) \\
        \theta =& L-\arctan \left(\frac{g}{f}\right)
    \end{aligned} \label{eqFormEquinoctial2}
\end{align}
式\eqref{eqFormEquinoctial1}\eqref{eqFormEquinoctial2}分别表示
开普勒转改进春分点和改进春分点转开普勒轨道根数。

% ////////////////////////////////////////////////////////////////
\subsection{轨道六根数和位置速度向量}
给定一个轨道物体的轨道六要素后,
即可计算出该物体相对于中心引力源的位置和速度向量,
换算关系为
\begin{align}
    R =& \left[\begin{matrix}
        c_{\Omega} c_{\omega}-s_{\Omega} c_{i} s_{\omega} & -c_{\Omega} s_{\omega}-s_{\Omega} c_{i} c_{\omega} & s_{\Omega} s_{i} \\
        s_{\Omega} c_{\omega}+c_{\Omega} c_{i} s_{\omega} & -s_{\Omega} s_{\omega}+c_{\Omega} c_{i} c_{\omega} & -c_{\Omega} s_{i} \\
        s_{i} s_{\omega} & s_{i} c_{\omega} & c_{i}
    \end{matrix}\right] \notag\\
    \vec{r} =& R\left[\begin{matrix}
        \frac{a(1-e^2)}{1+e\cos\theta}\cos\theta \\ \frac{a(1-e^2)}{1+e\cos\theta}\sin\theta \\ 0
    \end{matrix}\right] \notag\\
    \vec{v} =& R\left[\begin{matrix}
        -\sqrt{\frac{\mu}{a(1-e^2)}}\sin\theta \\ \sqrt{\frac{\mu}{a(1-e^2)}}(e+\cos\theta) \\ 0
    \end{matrix}\right] \label{eqFormEle2RV}\\
\end{align}
其中$\theta$为真近点角。
若已知初始时刻的轨道六要素,
求给定时刻的位置速度向量的步骤为
\begin{enumerate}[label={(\arabic*)}]\setlength{\itemsep}{-5pt}
    \item 根据轨道六要素求出平均角速度和初始平近点角;
    \item 根据初始平近点角、平均角速度、给定时刻求出给定时刻的平近点角;
    \item 由平近点角计算真近点角;
    \item 将给定时刻的真近点角代入式\eqref{eqFormEle2RV}计算位置速度向量;
\end{enumerate}
上述步骤需要平近点角和真近点角之间的换算。
由真近点角计算平近点角公式为
\begin{align*}
    E =& 2\arctan\left(\sqrt{\frac{1-e}{1+e}}\tan\frac{\theta}{2}\right) \\
    M =& E - e\sin{E}
\end{align*}
其中$E$为偏近点角,且与真近点角位于同一象限。
由平近点角$M$计算真近点角$\theta$没有解析解,
需要使用泰勒展开的方法,
具体公式如下\cite{msmart1977}
\begin{align*}
    \theta =& M+\left(2e-\frac{1}{4}e^3\right)\sin{M}
    + {\frac{5}{4}}e^2\sin{2M} \\
    &+ {\frac{13}{12}}e^3\sin{3M}+O(e^4)
\end{align*}

