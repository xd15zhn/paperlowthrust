%%%%%%%%%%%%%%%%%%%%%%%%%%%%%%%%%%%%%%%%%%%%%%%%%%%%%%%%%%%%%%%%
% Sim
%%%%%%%%%%%%%%%%%%%%%%%%%%%%%%%%%%%%%%%%%%%%%%%%%%%%%%%%%%%%%%%%
\section{仿真}
本文使用基于C++的仿真器\cite{olzhn2021}进行仿真。

% ////////////////////////////////////////
\subsection{给定参数计算}
给定出发点和到达点的改进春分点轨道根数分别为
\begin{center}\begin{tabular}{lll}
    \toprule
    名称 & 出发点 & 到达点 \\
    \midrule
    p & $1.00056767075005     $ & $1.51037523709971    $ \\
    f & $-0.00294437794279988 $ & $0.0854235624895203  $ \\
    g & $0.0162420106027165   $ & $-0.0378544516020575 $ \\
    h & $1.06259234267730e-05 $ & $0.0104735801650767  $ \\
    k & $3.28399488220299e-07 $ & $0.0122796628620782  $ \\
    L & $2.43306352346982     $ & $5.76521463465294    $ \\
    \bottomrule
\end{tabular}\end{center}
计算可得开普勒轨道根数为
\begin{center}\begin{tabular}{lll}
    \toprule
    名称 & 地球 & 火星 \\
    \midrule
    半长轴    a & $ 1.000840$ & $ 1.523677$ \\
    偏心率    e & $ 0.016507$ & $ 0.093435$ \\
    轨道倾角  i & $ 0.000021$ & $ 0.032276$ \\
    升交点赤经Ω & $ 0.030896$ & $ 0.864609$ \\
    近地点幅角ω & $-1.422358$ & $-1.281742$ \\
    真近点角    & $ 3.824526$ & $ 6.182348$ \\
    $\mu$值     & $398601   $ & $42808    $ \\
    \bottomrule
\end{tabular}\end{center}
另,太阳$\mu$值为$\mu_s=132706538114$。
其它设定参数为
\begin{center}\begin{tabular}{ll}
    \toprule
    名称 & 值 \\
    \midrule
    航天器质量 & 1000kg \\
    推力 & 1N \\
    比冲 & 2000s \\
    转移时间 & 260天 \\
    \bottomrule
\end{tabular}\end{center}
可算出以速度为单位的比冲为$I_{sp}=19600$m/s,
也就是说保持发动机推力$F=1$N时,
燃料消耗速度为$1/19600$千克/秒,
限制发动机工作时间不超过$T_f=19600*400=7.84\times10^6$秒,约90天。
为统一单位,将长度单位全部换算成km,
则推力为$F=10^-3$kN。

% ////////////////////////////////////////
\subsection{建模}
假设探测器在转移轨道上不受地球和火星引力影响。
考虑燃料消耗的火星探测器动力学方程为
\begin{align*}
    &\ddot{\vec{r}} = -\frac{\mu_s}{||\vec{r}||^3}\vec{r}
    + \frac{10^{-3}}{m(t)}\epsilon(t)
    \left[\begin{matrix}
        \cos\phi\cos\theta \\ \cos\phi\sin\theta \\ \sin\phi
    \end{matrix}\right] \\
    &\dot{m}(t) = -\frac{1}{19600}\epsilon(t)
\end{align*}
其中
$\phi$和$\theta$分别为发动机推力向量的俯仰角和方位角,
$\epsilon(t)=0$或$1$表示发动机开机或关机。
建立探测器器被控对象模型的模块框图如图所示。

% ////////////////////////////////////////
\subsection{绘制轨道}

