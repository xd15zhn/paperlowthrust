%%%%%%%%%%%%%%%%%%%%%%%%%%%%%%%%%%%%%%%%%%%%%%%%%%%%%%%%%%%%%%%%
% Sim
%%%%%%%%%%%%%%%%%%%%%%%%%%%%%%%%%%%%%%%%%%%%%%%%%%%%%%%%%%%%%%%%
\section{仿真}
本文使用基于C++的仿真器\cite{olzhn2021}进行仿真。

% ////////////////////////////////////////
\subsection{已知参数}
给定出发点和到达点的改进春分点轨道根数分别为
\begin{center}\begin{tabular}{lll}
    \toprule
    名称 & 出发点 & 到达点 \\
    \midrule
    p & $1.00056767075005     $ & $1.51037523709971    $ \\
    f & $-0.00294437794279988 $ & $0.0854235624895203  $ \\
    g & $0.0162420106027165   $ & $-0.0378544516020575 $ \\
    h & $1.06259234267730e-05 $ & $0.0104735801650767  $ \\
    k & $3.28399488220299e-07 $ & $0.0122796628620782  $ \\
    L & $2.43306352346982     $ & $5.76521463465294    $ \\
    \bottomrule
\end{tabular}\end{center}
其它设定参数为
\begin{center}\begin{tabular}{ll}
    \toprule
    名称 & 值 \\
    \midrule
    航天器质量 & 1000kg \\
    推力 & 1N \\
    比冲 & 2000s \\
    转移时间 & 260天 \\
    \bottomrule
\end{tabular}\end{center}
可算出以速度为单位的比冲为$I_{sp}=19600m/s$,
也就是说保持发动机推力$F=1$N时,
燃料消耗速度为$1/19600$千克/秒,
限制发动机工作时间不超过$T_f=19600*800=1.568\times10^7$秒。
为统一单位,将长度单位全部换算成km,
则推力为$10^-3$kN。
地球和火星的轨道根数和$mu$值为\cite{mhongbo2015}
\begin{center}\begin{tabular}{lll}
    \toprule
    名称 & 地球 & 火星 \\
    \midrule
    半长轴    a & $0.9999858 $ & $1.5237018$\\
    偏心率    e & $0.01667835$ & $0.09351464$\\
    轨道倾角  i & $0         $ & $1.849520$\\
    升交点赤经Ω & $0         $ & $49.67287$\\
    近地点幅角ω & $103.30275 $ & $336.27428$\\
    $\mu$值    & $398601     $ & $42808$\\
    \bottomrule
\end{tabular}\end{center}
太阳$\mu$值为$\mu_s=132706538114$。

% ////////////////////////////////////////
\subsection{建模}
考虑燃料消耗的火星探测器动力学方程为
\begin{align*}
    \ddot{\vec{r}} =& -\frac{\mu_s}{||\vec{r}||^3}\vec{r}
    - \frac{\mu_e}{||\vec{r}-\vec{r}_e||^3}(\vec{r}-\vec{r}_e) \\
    &- \frac{\mu_m}{||\vec{r}-\vec{r}_m||^3}(\vec{r}-\vec{r}_m)
    + \frac{10^-3}{m(t)}\epsilon(t) \\
    \dot{m}(t) =& -\frac{1}{19600}\epsilon(t)
\end{align*}
其中$\epsilon(t)=0$或$1$表示发动机开机或关机。
% \label{eqSimFA}

% ////////////////////////////////////////
\subsection{绘制轨道}

