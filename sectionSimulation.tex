%%%%%%%%%%%%%%%%%%%%%%%%%%%%%%%%%%%%%%%%%%%%%%%%%%%%%%%%%%%%%%%%
% Sim
%%%%%%%%%%%%%%%%%%%%%%%%%%%%%%%%%%%%%%%%%%%%%%%%%%%%%%%%%%%%%%%%
\section{仿真}
本文使用基于C++的仿真器\cite{olzhn2021}进行仿真。
给定出发点和到达点的改进春分点轨道根数分别为
\begin{center}\begin{tabular}{lll}
    \toprule
    名称 & 出发点 & 到达点 \\
    \midrule
    p & $1.00056767075005     $ & $1.51037523709971    $ \\
    f & $-0.00294437794279988 $ & $0.0854235624895203  $ \\
    g & $0.0162420106027165   $ & $-0.0378544516020575 $ \\
    h & $1.06259234267730e-05 $ & $0.0104735801650767  $ \\
    k & $3.28399488220299e-07 $ & $0.0122796628620782  $ \\
    L & $2.43306352346982     $ & $5.76521463465294    $ \\
    \bottomrule
\end{tabular}\end{center}
其它设定参数为
\begin{center}\begin{tabular}{ll}
    \toprule
    名称 & 值 \\
    \midrule
    航天器质量 & 1000kg \\
    推力 & 1N \\
    比冲 & 2000s \\
    转移时间 & 260天 \\
    \bottomrule
\end{tabular}\end{center}
可算出以速度为单位的比冲为$I_{sp}=19600m/s$,
也就是说保持发动机推力$F=1$N时,
燃料消耗速度为$1/19600$千克/秒。

% ////////////////////////////////////////
\subsection{仿真参数分析}
